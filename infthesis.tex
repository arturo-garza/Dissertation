%%%%%%%%%%%%%%%%%%%%%%%%
%% Sample use of the infthesis class to prepare a thesis. This can be used as 
%% a template to produce your own thesis.
%%
%% The title, abstract and so on are taken from Martin Reddy's csthesis class
%% documentation.
%%
%% MEF, October 2002
%%%%%%%%%%%%%%%%%%%%%%%%

%%%%
%% Load the class. Put any options that you want here (see the documentation
%% for the list of options). The following are samples for each type of
%% thesis:
%%
%% Note: you can also specify any of the following options:
%%  logo: put a University of Edinburgh logo onto the title page
%%  frontabs: put the abstract onto the title page
%%  deptreport: produce a title page that fits into a Computer Science
%%      departmental cover [not sure if this actually works]
%%  singlespacing, fullspacing, doublespacing: choose line spacing
%%  oneside, twoside: specify a one-sided or two-sided thesis
%%  10pt, 11pt, 12pt: choose a font size
%%  centrechapter, leftchapter, rightchapter: alignment of chapter headings
%%  sansheadings, normalheadings: headings and captions in sans-serif
%%      (default) or in the same font as the rest of the thesis
%%  [no]listsintoc: put list of figures/tables in table of contents (default:
%%      not)
%%  romanprepages, plainprepages: number the preliminary pages with Roman
%%      numerals (default) or consecutively with the rest of the thesis
%%  parskip: don't indent paragraphs, put a blank line between instead
%%  abbrevs: define a list of useful abbreviations (see documentation)
%%  draft: produce a single-spaced, double-sided thesis with narrow margins
%%
%% For a PhD thesis -- you must also specify a research institute:
%\documentclass[phd,ilcc,twoside]{infthesis}

%% For an MPhil thesis -- also needs an institute
% \documentclass[mphil,ianc]{infthesis}

%% MSc by Research, which also needs an institute
% \documentclass[mscres,irr]{infthesis}

%% Taught MSc -- specify a particular degree instead. If none is specified,
%% "MSc in Informatics" is used.
% \documentclass[msc,cogsci]{infthesis}
\documentclass[msc,logo,12pt,A4]{infthesis}  % for the MSc in Informatics

%% Master of Informatics (5 year degree)
% \documentclass[minf]{infthesis}

%% Undergraduate project -- specify the degree course and project type
%% separately
% \documentclass[bsc]{infthesis}
% \course{Artificial Intelligence and Psychology}
% \project{Fourth Year Project Report}

%% Put any \usepackage commands you want to use right here; the following is 
%% an example:
\usepackage{natbib}
% For subsections management
\usepackage{titlesec}
\usepackage{xcolor}
\usepackage{flafter}
\usepackage{hyperref}
\usepackage{apacite} 
\setcounter{secnumdepth}{4}

\titleformat{\paragraph}
{\normalfont\normalsize\bfseries}{\theparagraph}{1em}{}
\titlespacing*{\paragraph}
{0pt}{3.25ex plus 1ex minus .2ex}{1.5ex plus .2ex}

%% Information about the title, etc.
\title{Lexical Model for Low-resource Neural Machine Translation}
\author{Erick Arturo Garza Jacinto}

%% If the year of submission is not the current year, uncomment this line and 
%% specify it here:
% \submityear{1785}

%% Optionally, specify the graduation month and year:
 %\graduationdate{August 2018}

%% Specify the abstract here.
\abstract{%
 In recent years, there has been interest in neural machine translation (NMT) \citep*{DBLP:journals/corr/abs-1709-07809}. This approach to machine translation has shown good results even surpassing phrase based translation systems \citep*{DBLP:journals/corr/BritzGLL17}. Even though this approach have given promising results, they still show poor performance when they handle rare words \citep*{DBLP:journals/corr/abs-1709-07809}. These being words that have very few occurrences in the training data. This causes these models to require a lot of input data to provide good quality translations. 
  
  In the past couple of years, there have been approaches aiming to improve the translation quality of these models by improving the way they handle or generalize rare words. Some of these  approaches aim to do so by reducing the algorithms' need for data in order to learn meaningful information from it \citep*{DBLP:journals/corr/MiWI16}, while others attempt to better handle unknown words or rare words. On this second trend, \citet{DBLP:journals/corr/JeanCMB14} improve performance by allowing their architecture to look up rare words in a dictionary while \citet{DBLP:journals/corr/SennrichHB15} use word pieces obtained through algorithms such as Byte Pair Encoding (BPE) to try to cover a broader range of words taking into account their roots and morphology variations.
  
  %(\citealp{DBLP:journals/corr/JeanCMB14, DBLP:journals/corr/GulcehreANZB16, DBLP:journals/corr/LuongSLVZ14, DBLP:journals/corr/GuLLL16, DBLP:journals/corr/SennrichHB15}). From these second type of approaches, it is important to highlight two uses a dictionary to handle rare words 
  %presented a method to try to alleviate the huge amount of data the whole vocabulary represents by narrowing down the vocabulary to a sentence or batch scope size. 
  
  Considering previous work in the Neural Machine Translation field, I propose investigating the real learning improvement of the lexical model within a sequence-to-sequence model as presented by \citet*{DBLP:journals/corr/abs-1710-01329}. In contrast with other presented  works, the data this model needs does not require much pre-processing while, according to the authors, it still provides good translation improvements. This approach, has not been properly compared to standard approaches such as Byte Pair Encoding (BPE). Our hypothesis is that adding the Lexical model to the architecture as proposed by \citet{DBLP:journals/corr/abs-1710-01329} does help improve the translation quality. To prove this method's appropriateness, we will evaluate first, whether it helps when used in conjunction with previous approaches to rare words such as BPE, second, whether the lexicon model helps in low-resource situations and third, whether it allows us to leverage bilingual lexicons. This will be tested on three different architectures on which we will investigate if the Lexical Model provides an improvement. If the architecture shows the expected translation enhancement in contrast to the baseline architectures, this approach could be considered a novel machine translation architecture which in turn could help improve machine translation model's accuracy even in low-resource data conditions.
}

%% Now we start with the actual document.
\begin{document}

%% First, the preliminary pages
\begin{preliminary}

%% This creates the title page
\maketitle
%% Acknowledgements
\begin{acknowledgements}
I would like to thank my dissertation supervisor Lexi Birch for believing in me and guiding me through this process. Also I would like to thank my parents for supporting me all the time.
Finally, I would like to thank the State of Jalisco in Mexico for granting me the funding that enabled me to study at the University of Edinburgh.
\end{acknowledgements}

%% Next we need to have the declaration.
\standarddeclaration

%% Finally, a dedication (this is optional -- uncomment the following line if
%% you want one).
% \dedication{To my mummy.}

%% Create the table of contents
\tableofcontents
 
%\listoffigures
%\listoftables


%% If you want a list of figures or tables, uncomment the appropriate line(s)
% \listoffigures
% \listoftables

\end{preliminary}

%%%%%%%%
%% Include your chapter files here. See the sample chapter file for the basic
%% format.

%% Sample chapter file, for use in a thesis.
%% Don't forget to put the \chapter{...} header onto each file.

\chapter{Introduction}

This thesis examines the improvement gained in translation performance by introducing lexicon information into an Attention Based Neural Machine Translation Language Model \citep{DBLP:journals/corr/BahdanauCB14}. This is achieved by jointly training the previously mentioned architecture with a Feed-Forward neural network which we will call a Lexical model \citep{DBLP:journals/corr/abs-1710-01329}. 
This thesis focuses on the mentioned architecture because it is the current state of the art architecture for Machine Translation. This technique has shown superior results than those showed by the the previous state of the art phrase-based statistical machine translation techniques \citep{DBLP:journals/corr/abs-1709-07809}.  

Recurrent Neural Language Models provide several advantages from which the most important are: 1) They can relate words that appear in similar contexts as they use word embedding to represent them, thus words that are similar would have similar embedding. 2) They are able to represent long distance dependencies since their recurrent nature allows them to model sentences of arbitrary length. 

Conversely, the model also has several disadvantages: 1) They require a lot of training data to provide good results since they have a lot of trainable parameters. 2) They can not translate well out of vocabulary (OOV) words or words that appear only once in the training corpus (singleton).

Since Recurrent Neural Language Models became the most widely used technique, several works have been proposed to improve their performance. Some of these approaches aim to do so by designing architectures that learn meaningful information from the data \citep*{DBLP:journals/corr/MiWI16}, ultimately reducing the algorithms' need for data, while others attempt to better handle unknown words or rare words. In this thesis we use as our baseline two current standard practices from the second described type, sub-word unit representations which is used to augment input data \citep{DBLP:journals/corr/SennrichHB15} and bilingual pre-training proposed by \citet{DBLP:journals/corr/ZophYMK16}. With this experiments we seek to answer the following questions:

\begin{enumerate}
    \itemsep0em 
    \item Does training a Feed-Forward Neural Network jointly with the classic Encoder-Decoder architecture improves the translation quality? 
    \item If translation is indeed improved, is it because the model actually learns from individual words as claimed by the authors?
    \item How does the lexicon model interact with other techniques proposed to handle rare words, in particular with:
    \begin{itemize}
        \itemsep0em 
        \item Byte Pair Encoding (BPE) \citep{DBLP:journals/corr/SennrichHB15}
        \item Transfer Learning \citep{DBLP:journals/corr/ZophYMK16}
    \end{itemize}
\end{enumerate}


\section{Motivation}

Artificial Intelligence has shown strong performance when applied in several fields of human activities. For example, artificial intelligence applied to image recognition has been able to identify traffic signs with and error rate close to humans $\approx$ 0.2\% \citep*{DBLP:journals/corr/abs-1202-2745}. Similarly, we have seen computers beat world-champion chess player Garry
Kasparov in a six-game match in 1997 \citep{Campbell:2002:DB:512148.512152}.
Unlike in Image vision or playing chess, the field of Machine Translation has not yet seen this human-like performance. Nevertheless, Neural networks have provided a substantial performance boost to Machine translation; although they are still far from translations of human-like quality. 

The reason why machine translation has proven to be a more complicated problem than others yields in the fact that language is very complex. This means it can not be defined by sets of rules.\textcolor{red}{Add more}

Machine translation neural networks require large amounts of data to train in order to be able to generalise well. Additionally, gathering data is costly. Baring this in mind, algorithms that can generalise well with the least amount of data are desirable. 

Having as example how people learn a new language, initially translating only single words and subsequently translating more complex sentences. This thesis, looks at bilingual dictionaries or lexicons aims to prove the real gain obtained by adding a Feed Forward neural network proposed by \citet{DBLP:journals/corr/BahdanauCB14} to current state of the art techniques.

\section{Thesis Outline}

In Chapter \ref{ch:chapter2}, in the first place we present the literature review. In it, we describe the evolution of machine translation architectures with particular interest in the language model that we used in our experiments, the Recurrent Neural Network Language Model. Then, we describe the techniques we used as our baseline in the experimentation phase, presented by \citet{DBLP:journals/corr/SennrichHB15} and \citet{DBLP:journals/corr/ZophYMK16}. to address the translation problem we seek to address with the Lexical model. 

Chapter \ref{ch:chapter3} presents a thorough description of the lexical model. In section \ref{sec:datasets}, we describe the data sets used to train and evaluate the Language Model which include the TED data set \citep{TIEDEMANN12.463} and the news. In section \ref{sec:experimentdesign} we describe the experimental configuration used to incorporate the lexical model. Finally, we present the results from the experiments and their analysis.

Finally, Chapter \ref{ch:chapter4} presents the conclusions and describe future work.
\chapter{Background}\label{ch:chapter2}

Machine translation has been a field of interest since the beginning of computers.

Before neural network language models became widely used in machine translation, statistical machine translation using \textit{n-gram} models was the state of the art technique. During this period, systems achieved good translations \textcolor{red}{How good}.

Statistical machine translation has problems to generalise words that are not widely seen in the data it trains on as the probabilities it uses to predict the translation is based on the counts of the time a certain word has occurred alone and its co-occurrences with other words. To reduce this problem, smoothing techniques were introduced. 

Neural Networks for Machine Translation is a method first used by~\citet*{kalchbrenner13emnlp}, who proposed using convolutional \textit{n-gram} models to extract fixed-length vector of a source sentence \citep{DBLP:journals/corr/ChungGCB14}. Afterwards, the authors [\citealp{DBLP:journals/corr/SutskeverVL14,DBLP:journals/corr/ChoMGBSB14}] used an Recurrent Neural Network encoder and decoder architecture. The three mentioned works were the pioneers in the use of different types of neural networks at their core. They all have in common that as their core, they implement the now widely used Encoder Decoder Architecture.

Since their first appearance, Neural Network Language Models have increasingly proven great contribution to the field as they can achieve very competitive translations when training them with a large quantity of data for language pairs such as English-French ~\citep*{DBLP:journals/corr/ZophYMK16}. This is because Neural Network Language Models are useful to model conditional probability distributions with several inputs \citep{DBLP:journals/corr/abs-1709-07809}. 

One of the questions that aroused when neural networks started to be used was how should words be represented to feed them into the neural network. The answer was presented by \citet*{DBLP:journals/corr/abs-1301-3781} and later improved in a subsequent paper \citep{DBLP:journals/corr/MikolovSCCD13}. They proposed representing words as a continuous vector representation. The proposed vector representation, provided state-of-the-art performance for measuring syntactic and semantic word similarities \citep{DBLP:journals/corr/abs-1301-3781}.

Similar to previous Statistical Machine Translation language models, Neural Network models tend to have problems to translate rare or unseen word.

This problem is more evide

It soon became clear that to obtain better translation quality with less data, that we needed 

 

As basic unit cells of modern Recurrent Neural Network Language Models there have been two proposals which are widely used nowadays as cells of the RNN. First, the Long Short Term Memory (LSTM) unit cell proposed by \citet{hochreiter1997long}. Its purpose is to alleviate the problem of vanishing or exploding gradients by . Second, the Gated Recurrent Units proposed by \citet*{DBLP:journals/corr/ChoMGBSB14} aims to .

In this work we use Gated Recurrent Units to conform with the Nematus Framework implementation. These units are defined as:

\textcolor{red}{Insert definition of GRU}
\citep{DBLP:journals/corr/ChungGCB14}

Having defined the general architecture of an Attention Based Recurrent Neural Network, here are some of the works that have been presented to alleviate the poor generalisation when in low resource conditions. The work presented by \citet*{DBLP:journals/corr/JeanCMB14}, tries to reduce the amount of unknown words by allowing the system to look up words in a dictionary. A disadvantage of this approach is that since we cannot have all the possible words that can happen in new data in a dictionary, then a way to handle the words that the system did not find needs to be put in place. This is exactly what the work done by \citet*{DBLP:journals/corr/GulcehreANZB16}, \citet*{DBLP:journals/corr/LuongSLVZ14} and \citet*{DBLP:journals/corr/GuLLL16} aim to solve.
When a word is unknown, their approach simply replaces them with UNK (unknown) tokens. The process is carried out in two stages, during the encoding process, words that are not known are replaced with UNK tokens. Then during the decoding stage, the UNK tokens are replaced back by copying the original word in the source text to the target one. While this is a sufficiently good approach for proper names like company names as the do not change from one language to the other, it is has encountered problems as it has been observed that the UNKs are replaced by incorrect words in the target text. While this approach was a good start to tackle the rare-word problem, it has no one word to one word equivalence and it is unnecessarily complex to have to train a separate dictionary. On a similar path to the work presented by \citet{DBLP:journals/corr/JeanCMB14}, \citet*{DBLP:journals/corr/MiWI16} presented a method that improves translation by narrowing down the vocabulary to a sentence or batch scope vocabulary size. To generate the reduced size vocabularies, \citet{DBLP:journals/corr/MiWI16} use a classical Statistical Machine Translation (SMT) system to build a word-to-word and a phrase-to-phrase translation models. This vocabulary reduction decreases the amount of UNK words by allowing the system to explore a larger vocabulary \citep{DBLP:journals/corr/abs-1710-01329}. While this improves the BLEU score ~\citep{Papineni02bleu:a} from their baseline by one point \citep{DBLP:journals/corr/MiWI16}. This method has as a drawback that it requires a separate model to be trained and this model still requires a large amount of data to achieve a good translation quality. As expected, the previous approaches do yield improvement in the translation score, nevertheless, they only address unknown words.

A second approach to this problem was followed by \citet*{DBLP:journals/corr/SennrichHB15} and \citet*{DBLP:journals/corr/ArthurNN16}. The approach presented by \citet{DBLP:journals/corr/SennrichHB15} is language independent, simple and efficient. It has as drawback that the segmentations it performs are not linguistically motivated which sometimes yield non optimal splits and that it still struggles with low count events. The approach presented by \citet{DBLP:journals/corr/ArthurNN16} attempts to address the rare word problem by gaining more information from the data. They learn information from the data by simply incorporating discrete, probabilistic (count based) lexicons as an additional source of information in the Neural Machine Translation system.  Their results do yield improvements in contrast to previous work. However, their proposed lexicon must be trained separately in an Stochastic Machine Translation (SMT) system, and its parameters can be difficult to handle in GPU memory \citep{DBLP:journals/corr/abs-1710-01329}. With this approach, \citet{DBLP:journals/corr/ArthurNN16} achieved an increase in BLEU of 2.0-2.3 points and 0.13-0.44 points in the NIST. Additionally, they observed qualitative improvements in the translations of content words. Even though these results are promising, the extra training step makes their approach too complex to train. Their work is the most similar to the one presented by \citet{DBLP:journals/corr/abs-1710-01329}.

A powerful idea is Transfer Learning, a technique published by~\citet*{DBLP:journals/corr/ZophYMK16}. This technique consists in using knowledge from a high-resource language pairs such as French-English to better translate low-resource language-pairs. The idea is that with the training step carried out on the high-resource language we improve the performance the system achieves to  the related task, typically reducing the amount of data required ~\citep*{DBLP:journals/corr/ZophYMK16}.

An interesting discovery made by \citet{DBLP:journals/corr/ZophYMK16} is that if the network is trained with a high-resource language pair that is related with the low-resource one, then the system obtained better performance.
Inspired by this technique, \citet*{DBLP:journals/corr/abs-1710-04087} proposed initialising with monolingual data. This approach is simpler than the one presented by \citet{DBLP:journals/corr/ZophYMK16} as it does not require a parallel corpus. In this theses we

Character level



%A second approach to this problem was followed by \citet*{DBLP:journals/corr/SennrichHB15} and \citet*{DBLP:journals/corr/ArthurNN16}. The approach presented by \citet{DBLP:journals/corr/SennrichHB15} is language independent, simple and efficient. It has as drawback that the segmentations it performs are not linguistically motivated which sometimes yield non optimal splits and that it still struggles with low count events. The approach presented by \citet{DBLP:journals/corr/ArthurNN16} attempts to address the rare word problem by gaining more information from the data. They learn information from the data by simply incorporating discrete, probabilistic (count based) lexicons as an additional source of information in the Neural Machine Translation system.  Their results do yield improvements in contrast to previous work. However, their proposed lexicon must be trained separately in an Stochastic Machine Translation (SMT) system, and its parameters can be difficult to handle in GPU memory \citep{DBLP:journals/corr/abs-1710-01329}. With this approach, \citet{DBLP:journals/corr/ArthurNN16} achieved an increase in BLEU of 2.0-2.3 points and 0.13-0.44 points in the NIST. Additionally, they observed qualitative improvements in the translations of content words. Even though these results are promising, the extra training step makes their approach too complex to train. Their work is the most similar to the one presented by \citet{DBLP:journals/corr/abs-1710-01329}. This is why we believe the simple Feed Forward Neural network that constitutes the Lexical model is a better approach as it does not need a separate training.
\chapter{Lexical Model for Low Resource Languages} \label{ch:chapter3}
\section{Introduction}

Natural Languages frequently a sparse distribution of words as they tend have few words that appear more and others that do not. This is called a power law distribution. 

Statistical machine translation learns from the data that it is presented to. Distributions such as the present in natural languages, creates a knowledge void that is not easily filled.

\section{The Lexical Model}
The lexicon model was introduced by \citet{DBLP:journals/corr/abs-1710-01329}. This model consists in training a Feed Forward Network in parallel with an Attention Based Recurrent Neural Network.

Their implementation is

\section{Experimental design}\label{sec:experimentdesign}

\subsection{Baselines} \label{sec:baselines}

To be able to compare the lexical model with a robust baseline we selected a range of techniques widely used to improve performance. The selected techniques were:


\subsubsection{Lexical model with small dataset}

As a starting point we evaluated the performance of the Lexical model against a our baseline using a small data set. The 

\subsubsection{Lexical model and a transfer learning with a bilingual dictionary}

\subsubsection{Lexical model and a transfer learning with monolingual data}

\subsection{Datasets} \label{sec:datasets}

The datasets used for this dissertation are the TED english-german datased and the WMT17 English - German Task. We used as our validation data set the News-crawl 2013 and our test data set the News-crawl 2017.

\subsubsection{TED talks parallel corpus}

\subsubsection{WMT17 corpus}

\subsubsection{WMT17 corpus} \citep*{bojar2017d1}

\section{Experimental Results}


\section{Analysis}
\chapter{Conclusion and Future Work} \label{ch:chapter4}
\section{Summary and Conclusion}
\section{Future Work}
%\chapter{Improvements to the Lexical Model}
%% ... etc ...

%%%%%%%%
%% Any appendices should go here. The appendix files should look just like the
%% chapter files.
%\appendix
%\chapter{Aliquam erat volutpat}

(If you're wondering what all this weirdness is, check out\\
http://www.subterrane.com/loremipsum.shtml)

Aliquam erat volutpat. Phasellus sed tortor at metus luctus venenatis.
Etiam vel dolor vel lectus elementum adipiscing. Donec sit amet dolor. In
hac habitasse platea dictumst. Nullam bibendum. Etiam eget mauris non velit
faucibus volutpat. Ut ultrices nonummy mi. Praesent convallis, tellus eget
posuere auctor, est est mollis risus, vitae fringilla orci nisl vel erat.
Morbi ultricies. Proin consequat. Praesent consequat nulla a mauris.
Vivamus tellus. 

\section{Proin consequat}

Sed blandit nunc id massa. Integer dictum euismod tellus. Sed metus nunc,
rhoncus ut, volutpat in, lacinia ac, dolor. Vestibulum quis augue
vel dui volutpat eleifend. Praesent vulputate mattis leo. Phasellus pretium
semper libero. Mauris a enim non pede convallis suscipit. Suspendisse nibh
diam, luctus in, cursus at, dignissim nec, pede. Aenean semper massa.
Pellentesque habitant morbi tristique senectus et netus et malesuada fames ac
turpis egestas. Nullam a pede ut ligula viverra vehicula. Sed augue mi,
rhoncus ut, ultrices sed, tincidunt eget, libero. Nam quis dolor. Nunc
fermentum hendrerit arcu. Integer non enim. Aenean blandit velit et felis. Cum
sociis natoque penatibus et magnis dis parturient montes, nascetur ridiculus
mus. Curabitur nonummy malesuada pede. Nunc et enim. Quisque dui. Pellentesque
in felis. 

Sed id mi. Pellentesque pede leo, auctor et, interdum eu, posuere semper,
nisl. Morbi commodo euismod wisi. Cras ornare mauris et erat. Duis neque
neque, pretium ut, bibendum nec, ultrices a, lacus. Nullam lobortis. Ut
luctus, diam non tempus pellentesque, dui justo consequat ligula, eu
consectetuer tortor diam vel dui. 

\begin{figure}
    \begin{center}
        A figure.
        \caption{Nunc lacinia}
    \end{center}
\end{figure}

Phasellus porttitor. In pede lacus, convallis semper, fermentum eu,
vehicula quis, dui. Ut sodales pede sed est. Cras lacinia. Nulla ac augue
in lectus sodales ultricies. Nam velit nunc, convallis ac, ullamcorper
semper, malesuada vel, eros. Nunc risus. Vestibulum ac erat. Sed id justo
id nibh viverra facilisis. Curabitur laoreet. Nunc sodales odio at mauris.
Vestibulum tincidunt sem eget pede. Nulla nec risus non wisi varius porta.
Morbi nibh. Donec lacus. Vestibulum ante ipsum primis in faucibus orci
luctus et ultrices posuere cubilia Curae; Vestibulum lectus. Suspendisse
sed dui. 

Nunc lacinia, sapien nec fermentum pretium, turpis elit egestas metus, a
interdum tellus justo semper neque. Integer quis purus semper metus
vestibulum pharetra. Maecenas commodo fermentum wisi. Pellentesque diam.
Proin sit amet orci. Praesent auctor. Sed tortor. Sed sodales aliquam diam.
Vivamus cursus leo nec velit. Sed non pede. Nulla tempor imperdiet est.
Curabitur ornare cursus ante. Sed varius lobortis quam. Quisque ac arcu id
wisi ultrices pellentesque. Pellentesque eleifend consequat ipsum. Fusce
vestibulum sagittis lectus. Fusce risus. Duis felis. Suspendisse justo.
Integer ut libero a purus egestas luctus. Mauris dictum augue a enim.
Vivamus sodales placerat ipsum. Nunc lacus. 

\begin{quote}
Curabitur dictum. Donec vestibulum diam nec lacus. Nulla convallis, eros
vitae varius volutpat, erat quam facilisis purus, in accumsan dolor felis
vitae nunc. 
\end{quote}

Aliquam erat volutpat. Phasellus sed tortor at metus luctus venenatis.
Etiam vel dolor vel lectus elementum adipiscing. Donec sit amet dolor. In
hac habitasse platea dictumst. Nullam bibendum. Etiam eget mauris non velit
faucibus volutpat. \footnote{Ut ultrices nonummy mi. Praesent convallis, tellus eget
posuere auctor, est est mollis risus, vitae fringilla orci nisl vel erat.
Morbi ultricies.} Proin consequat. Praesent consequat nulla a mauris.
Vivamus tellus. 

Phasellus porttitor. In pede lacus, convallis semper, fermentum eu,
vehicula quis, dui. Ut sodales pede sed est. Cras lacinia. Nulla ac augue
in lectus sodales ultricies. Nam velit nunc, convallis ac, ullamcorper
semper, malesuada vel, eros. Nunc risus. Vestibulum ac erat. Sed id justo
id nibh viverra facilisis. Curabitur laoreet. Nunc sodales odio at mauris.
Vestibulum tincidunt sem eget pede. Nulla nec risus non wisi varius porta.
Morbi nibh. Donec lacus. Vestibulum ante ipsum primis in faucibus orci
luctus et ultrices posuere cubilia Curae; Vestibulum lectus. Suspendisse
sed dui. 

Sed id mi. Pellentesque pede leo, auctor et, interdum eu, posuere semper,
nisl. Morbi commodo euismod wisi. Cras ornare mauris et erat. Duis neque
neque, pretium ut, bibendum nec, ultrices a, lacus. Nullam lobortis. Ut
luctus, diam non tempus pellentesque, dui justo consequat ligula, eu
consectetuer tortor diam vel dui. 


%% ... etc...

%% Choose your favourite bibliography style here.
\bibliographystyle{apalike}

%% If you want the bibliography single-spaced (which is allowed), uncomment
%% the next line.
% \singlespace

%% Specify the bibliography file. Default is thesis.bib.
\bibliography{thesis}

%% ... that's all, folks!
\end{document}
